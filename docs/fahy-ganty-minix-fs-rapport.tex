              
%%%%%%%%%%%%%%%%%%%%%%%%%%%%%%%%%%%%%%%%%%%%%%%%%%%%%%%%%%%%%%%%%%%%%
% LaTeX Template: Project Titlepage Modified (v 0.1) by rcx
%
% Original Source: http://www.howtotex.com
% Date: February 2014
% 
% This is a title page template which be used for articles & reports.
% 
% This is the modified version of the original Latex template from
% aforementioned website.
% 
%%%%%%%%%%%%%%%%%%%%%%%%%%%%%%%%%%%%%%%%%%%%%%%%%%%%%%%%%%%%%%%%%%%%%%

% 12.06.2015
% Axel Fahy

\documentclass[12pt]{article}
\usepackage[a4paper]{geometry}
\usepackage[utf8]{inputenc}  
\usepackage[myheadings]{fullpage}
\usepackage{fancyhdr}
\usepackage{lastpage}
\usepackage{graphicx, wrapfig, subcaption, setspace, booktabs}
\usepackage[T1]{fontenc}
\usepackage[font=small, labelfont=bf]{caption}
\usepackage{fourier}
\usepackage[protrusion=true, expansion=true]{microtype}
\usepackage[english]{babel}
\usepackage{sectsty}
\usepackage{url, lipsum}


\newcommand{\HRule}[1]{\rule{\linewidth}{#1}}
\onehalfspacing
\setcounter{tocdepth}{5}
\setcounter{secnumdepth}{5}


% Remove the 'Chapter' before each chapter.
%\usepackage{titlesec}
%\titleformat{\chapter}{}{}{0em}{\bf\LARGE} 

%\makeatletter
%\renewcommand{\@makechapterhead}[1]{%
%\vspace*{0 pt}%
%{\setlength{\parindent}{0pt} \raggedright \normalfont
%\bfseries\Huge\thechapter.\ #1
%\par\nobreak\vspace{40 pt}}}
%\makeatother

%-------------------------------------------------------------------------------
% HEADER & FOOTER
%-------------------------------------------------------------------------------
\pagestyle{fancy}
\fancyhf{}
\setlength\headheight{15pt}
\fancyhead[L]{}
\fancyhead[R]{\textsl{Système d'exploitation - Minix-FS}}
\fancyfoot[R]{\textsl{Page \thepage\ of \pageref{LastPage}}}
%-------------------------------------------------------------------------------
% TITLE PAGE
%-------------------------------------------------------------------------------

\begin{document}

\title{ \normalsize \textsc{Système d'exploitation}
        \\ [2.0cm]
        \HRule{0.5pt} \\
        \LARGE \textbf{\uppercase{Minix-FS}}
        \HRule{2pt} \\ [0.5cm]
        \normalsize \today \vspace*{5\baselineskip}}

\date{}

\author{
        Axel Fahy - Benjamin Ganty\\
        hepia - ITI}

\clearpage
\maketitle
\thispagestyle{empty} % Remove the number on page title.
\newpage
\tableofcontents
\newpage

%-------------------------------------------------------------------------------
% Section title formatting
\sectionfont{\scshape}
%-------------------------------------------------------------------------------

%-------------------------------------------------------------------------------
% BODY
%-------------------------------------------------------------------------------

\section{Introduction}

L'objectif de ce projet est d'implémenter un système de fichier MINIX accessible en 
lecture et en écriture depuis un fichier image formaté au format MINIX version 1.
MINIX est basé sur le système de fichiers UNIX, sans les aspects complexe de ce dernier.
\\\\
Le projet a été séparé en deux partie. La première consistait à manipuler les structures
de données de l'image, récupérée au préalable par des lectures de blocs, a été effectuée
en Python 2 sur la base du canevas des classes fournis avec le projet.
\\\\
La deuxième partie consistait à implémenter un accès distant sur ce système de fichier.
Pour ceci, le client (Python) se connecte à un serveur de bloc écrit en langage C.

\subsection{Répartition des tâches}

Pour le projet, Axel Fahy c'est occupé de la partie Python (sans les sockets).
\\
Quant à lui, Benjamin Ganty a implémenté le serveur en C ainsi que la partie client en 
Python.
Pour synchroniser les différentes modifications, nous avons utilisé un dépôt git.


\section{État du projet}
Le projet est fonctionnel et réussi à exécuter le \textit{tester} pour le Python ainsi
que pour la partie avec le serveur C. 
Les tests ont été effectués sur une machine, en utilisant l'adresse ip \textit{127.0.0.1} 
pour la connexion.

  \subsection{Python}

  \subsection{C}


\section{Difficultés rencontrées}

Concernant la première partie, j'ai eu de la peine avec la méthode \textit{ialloc\_bloc()}
, car cette fonction n'est pas entièrement expliquée dans les documents du cours. 
Par ailleurs, les tests unitaires ne vérifiait pas le fonctionnement de la méthode 
dans le cas d'un bloc indirect ou d'un bloc doublement indirect.
C'est pourquoi, je ne suis pas sûr de son comportement, plus particulièrement dans 
le cas de blocs doublement indirects.
\\\\
Les fichiers de tests mis à disposition ont créé beaucoup de problèmes pour
vérifier le fonctionnement de la manipulation des structures de données via
le serveur de blocs. Il n'était précisé nulle part que les fichiers de tests
rechargeais l'image de base pour les tests d'après. Les fichiers de tests ne pouvait
donc pas fonctionner avec un serveur distant pour ces raisons. C'est pour cela
que nous avons du modifier ces fichiers de tests de manière à effectué une copie
de sauvegarde de l'image courante pour pouvoir la réinitialiser ensuite.
Il aurait du y avoir un test unitaire à disposition spécialement prévu pour 
tester la communication avec les sockets.


\section{Conclusion}

En conclusion, ce projet nous a permis de découvrir le fonctionnement d'un système
de fichier. Cela a permis de donnée un aspect pratique à la théorie étudiée 
durant ce semestre. Il nous a également donnée l'opportunité de découvrir un
nouveau langage, le Python.

\end{document}
